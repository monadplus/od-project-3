
%%%% Encodings

\usepackage[utf8]{inputenc} % encoding
\usepackage[english]{babel} % use special characters and also translates some elements within the document.

%%%% Misc

\usepackage{hyperref}[colorlinks=true, urlclor=blue]       % Hyperlinks \url{url} or \href{url}{name}
\usepackage{parskip}        % \par starts on left (not idented)
\usepackage{tocbibind}      % Adds the bibliography to the table of contents (automatically)

% \usepackage[document]{ragged2e}  % Left-aligned (whole document)
% \begin{...} ... \end{...}   flushleft, flushright, center

%%%% Abstract

\usepackage{abstract}       % Abstract

% http://www.ctex.org/documents/packages/special/abstract.pdf
\renewcommand{\absnamepos}{flushleft} % \begin{abstract} \noindent ... \end{abstract}
\setlength{\absleftindent}{0pt}
\setlength{\absrightindent}{0pt}

%%%% Graphics

\usepackage{graphicx}
\graphicspath{{./figures/}} % directory to look up for graphics

% \begin{figure}[h]
%   \centering
%   \includegraphics[scale=0.5]{cat}  % [width=\textwidth, height=4cm],
%   \caption{Example of a cat}
%   \label{fig:cat}
% \end{figure}

%%%% Math

\usepackage{amsmath}        % Math
\usepackage{amssymb}        % New symbols http://milde.users.sourceforge.net/LUCR/Math/mathpackages/amssymb-symbols.pdf
\usepackage{bm}             % $\bm{D + C}$

\usepackage{amsthm} % Math, \newtheorem, \proof, etc
% \begin{theorem}\label{t:label}  ...  \end{theorem}
% \begin{proof} ... \end{proof}
\theoremstyle{plain} % default
\newtheorem{theorem}{Theorem}[section]
\newtheorem{corollary}{Corollary}[theorem]  % Numering depends on the current section (instead of global)
\newtheorem{lemma}[theorem]{Lemma} % Shares numeration with theorem.
\theoremstyle{definition}
\newtheorem{definition}{Definition}[section]
\theoremstyle{remark}
\newtheorem*{remark}{Remark}

% Defines a new environment to write your or claim - proof
\newenvironment{claim}[1]{\par\noindent\underline{Claim:}\space#1}{}
\newenvironment{claimproof}[1]{\par\noindent\underline{Proof:}\space#1}{\hfill $\blacksquare$}

%%%% Code/Pseudo-code

\usepackage{minted} % Code listing
% \mint{html}|<h2>Something <b>here</b></h2>|
% \inputminted{octave}{BitXorMatrix.m}

%\begin{listing}[H]
  %\begin{minted}[xleftmargin=20pt,linenos,bgcolor=codegray]{haskell}
  %\end{minted}
  %\caption{Example of a listing.}
  %\label{lst:example} % You can reference it by \ref{lst:example}
%\end{listing}

\newcommand{\code}[1]{\texttt{#1}} % Define \code{foo.hs} environment

\usepackage[vlined,ruled]{algorithm2e} % pseudo-code http://tug.ctan.org/macros/latex/contrib/algorithm2e/doc/algorithm2e.pdf

%%%% Colors

\usepackage{xcolor}         % Colours \definecolor, \color{codegray}
\definecolor{codegray}{rgb}{0.9, 0.9, 0.9}
% \color{codegray} ... ...
% \textcolor{red}{easily}

%%%% Math

%\makeglossaries % before entries

%\newglossaryentry{latex}{
    %name=latex,
    %description={Is a mark up language specially suited
    %for scientific documents}
%}

% Referene to a glossary \gls{latex}
% Print glossaries \printglossaries

\usepackage[acronym]{glossaries} %

% \acrshort{name}
% \acrfull{name}
% \newacronym{foo}{arcshort}{acrfull}

\usepackage{enumitem} % \begin{enumerate}[label=(\alph*)]

\usepackage{multirow}
